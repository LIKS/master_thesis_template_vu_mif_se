\documentclass[
    % english, % thesis will be in english
    % signatureplaces % adds signature places in the title page
]{VUMIFSEMasterThesis}
\usepackage{float}
\usepackage{wrapfig2}
\usepackage{hyperref}
\usepackage{algorithmicx}
\usepackage{algorithm}
\usepackage{algpseudocode}
\usepackage{amsfonts}
\usepackage{amsmath}
\usepackage{bm}
\usepackage{caption}
\usepackage{color}
\usepackage{graphicx}
\usepackage{listings}
\usepackage{subcaption}
\usepackage{biblatex}

% Title page
\university{Vilniaus universitetas}
\faculty{Matematikos ir informatikos fakultetas}
\department{Programų sistemų studijų programa}
\papertype{Magistro baigiamojo darbo projektas / Magistro baigiamasis darbas} % Leave one
\title{Programų sistemų kūrimo metodų tyrimas}
\titleineng{Investigation of Methods of Software Development}
\author{Vardenis Pavardenis}
% \secondauthor{Vardonis Pavardonis}   % Add second author
\supervisor{prof. habil. dr. Vardaitis Pavardaitis}
\reviewer{doc. dr. Vardauskas Pavardauskas}
\date{Vilnius – \the\year}

\bibliography{bibliography}

\begin{document}
\maketitle

%% Acknoledgements of individuals and/or institutions.
% \sectionnonumnocontent{}
% \vspace{7cm}
% \begin{center}
%     Padėkos asmenims ir/ar organizacijoms
% \end{center}

\begin{lithuanian}
\sectionnonumnocontent{Santrauka}
Santraukose lietuvių ir anglų kalbomis glaustai aprašomas darbo turinys:
pristatoma nagrinėta problema ir padarytos išvados. Santraukų gale nurodomi
darbo raktiniai žodžiai. Santraukos lietuvių ir anglų kalbomis rašomos
atskiruose puslapiuose. Kiekvienos jų apimtis -- ne daugiau kaip 0,5 puslapio.
Automatiškai naudojamos lietuviškos kabutės: \enquote{tekstas}.

% Specify up to 5 most important topic keywords.
% A keyword can have multiple words.
\raktiniaizodziai{raktinis žodis 1, raktinis žodis 2, raktinis žodis 3, raktinis žodis 4, raktinis žodis 5}
\end{lithuanian}

\begin{english}
\sectionnonumnocontent{Summary}
English summary. English quotes are used automatically: \enquote{tekstas}.

\keywords{keyword 1, keyword 2, keyword 3, keyword 4, keyword 5}
\end{english}

\tableofcontents

\sectionnonum{Įvadas}
Įvade aprašomi darbo tikslai ir uždaviniai, nurodomas temos aktualumas, aptariamos
teorinės darbo prielaidos bei metodika, apibrėžiamas tiriamasis objektas, apibūdinami
su tema susiję literatūros ar kitokie šaltiniai, temos analizės tvarka, darbo atlikimo
aplinkybės, pateikiama žinių apie naudojamus instrumentus (programas ir kt.). Darbo
įvadas neturi būti dėstymo santrauka. Įvado apimtis 3--4 puslapiai.

\section{Medžiagos darbo tema dėstymo skyriai}
Medžiagos darbo tema dėstymo skyriuose išsamiai pateikiamos nagrinėjamos temos detalės:
pradiniai duomenys, jų analizės ir apdorojimo metodai, sprendimų įgyvendinimas, gautų
rezultatų apibendrinimas. Šios dalies turinys labai priklauso nuo darbo temos. Tačiau
visais atvejais joje turi būti tokio pobūdžio skyriai:
\begin{enumerate}
\item literatūros ar kitokių šaltinių apžvalga. Čia reikėtų daugiau dėmesio skirti
nuodugnesnėms tam tikros srities studijoms, akademiniams straipsniams, įvairių autorių
nuomonių palyginimui;
\item analitinė dalis. Šiame skyriuje, nagrinėjant pasirinktą temą ir sprendžiant iškeltas
problemas, parenkami tyrimo metodai, kurie atitiktų ne tik temos pobūdį, bet ir
objektyvias tyrėjų galimybes. Autorius turi atlikti pakankamai išsamią ir temos turinį
atskleidžiančią savarankišką analizę;
\item objekto projektavimas. Šiame skyriuje, integruojant teorines bei praktines žinias,
aptariamos ir vertinamos galimos sprendimų alternatyvos, atskleidžiamas autoriaus 
siūlomas problemos sprendimo kelias, pateikiamas veiksmų planas ar bendrosios jų
atlikimo gairės;
\item objekto realizacija. Aprašoma objekto prototipo realizacija, jo savybės, pritaikymo
praktikoje galimybės.
\item verifikavimas ir vertinimas. Šiame skyriuje pateikiami argumentai parodantys
pasiūlyto sprendimo korektiškumą ir efektyvumą.
\end{enumerate}
Skyriai gali turėti poskyrius ir smulkesnes sudėtines dalis, kaip punktus ir papunkčius.

Medžiaga turi būti dėstoma aiškiai, pateikiant argumentus. Tekste dėstomas
trečiuoju asmeniu, t.y. rašoma ne \enquote{aš manau}, bet „autorius mano“, „autoriaus
nuomone“. Reikėtų vengti informacijos nesuteikiančių frazių, pvz., „...kaip jau
buvo minėta...“, \enquote{...kaip visiems žinoma...} ir pan., vengti grožinės
literatūros ar publicistinio stiliaus, gausių metaforų ar panašių meninės
išraiškos priemonių.

Skyriai gali turėti poskyrius ir smulkesnes sudėtines dalis, kaip punktus ir
papunkčius.

\subsection{Poskyris}
Citavimo pavyzdžiai: cituojamas vienas šaltinis \cite{PvzStraipsnLt}; cituojami
keli šaltiniai \cite{PvzStraipsnEn, ArticleByOrg, PvzKonfLt, PvzKonfEn, PvzKnygLt, PvzKnygEn,
PvzElPubLt, PvzElPubEn, PvzBakLt, PvzMagistrLt, PvzPhdEn}.

Anglų kalbos terminų pateikimo pavyzdžiai: priklausomybių injekcija (\anglnb{dependency injection},
dažnai trumpinama kaip \textit{DI}), saitų redaktorius \angl{linker}.

Išnašų\footnote{Pirma išnaša.} pavyzdžiai\footnote{Antra išnaša.}.

\subsection{Faktorialo algoritmas}

\ref{alg:factorial} algoritmas parodo, kaip suskaičiuoti skaičiaus faktorialą.

\begin{algorithm}
\caption{Skaičiaus faktorialas}
\begin{algorithmic}[1] % [1] adds line numbers
\State $N\gets$ skaičius, kurio faktorialą skaičiuojame
\State $F\gets 1$
\For{$i := 2$ to $N$}
    \State $F\gets F \cdot i$
\EndFor
\end{algorithmic}
\label{alg:factorial}
\end{algorithm}

\subsubsection{Punktas}
\subsubsubsection{Papunktis}
\subsubsection{Punktas}
\section{Skyrius}
\subsection{Poskyris}
\subsection{Poskyris}

\sectionnonum{Rezultatai}
Rezultatų skyriuje išdėstomi pagrindiniai darbo rezultatai: kažkas išanalizuota,
kažkas sukurta, kažkas įdiegta. Tarpinių žingsnių išdavos skirtos užtikrinti galutinio
rezultato kokybę neturi būti pateikiami šiame skyriuje. Kalbant informatikos termi-
nais, šiame skyriuje pateikiama darbo išvestis, kuri gali būti įvestimi kituose panašios
tematikos darbuose. Rezultatai pateikiami sunumeruotų (gali būti hierarchiniai) sąrašų
pavidalu. Darbo rezultatai turi atitikti darbo tikslą.

\sectionnonum{Išvados}
\begin{enumerate}[labelindent=0pt]
    \item Išvadų skyriuje daromi nagrinėtų problemų sprendimo metodų palyginimai, siūlomos
rekomendacijos, akcentuojamos naujovės.
    \item Išvados pateikiamos sunumeruoto (gali būti hierarchinis) sąrašo pavidalu.
    \item Darbo išvados turi atitikti darbo tikslą.
\end{enumerate}

\printbibliography[heading=bibintoc]  % Šaltinių sąraše nurodoma panaudota
% literatūra, kitokie šaltiniai. Abėcėlės tvarka išdėstomi darbe panaudotų
% (cituotų, perfrazuotų ar bent paminėtų) mokslo leidinių, kitokių publikacijų
% bibliografiniai aprašai. Šaltinių sąrašas spausdinamas iš naujo puslapio.
% Aprašai pateikiami netransliteruoti. Šaltinių sąraše negali būti tokių
% šaltinių, kurie nebuvo paminėti tekste (LaTeX tai sutvarko automatiškai).
% Šaltinių sąraše rekomenduojame necituoti savo kursinio darbo, nes tai nėra
% oficialus literatūros šaltinis. Jei tokių nuorodų reikia, pateikti jas tekste.

% \sectionnonum{Sąvokų apibrėžimai}
\sectionnonum{Santrumpos}
Sąvokų apibrėžimai ir santrumpų sąrašas sudaromas tada, kai darbo tekste
vartojami specialūs paaiškinimo reikalaujantys terminai ir rečiau sutinkamos
santrumpos.

% Appendices
% Prieduose gali būti pateikiama pagalbinė, ypač darbo autoriaus savarankiškai
% parengta, medžiaga. Savarankiški priedai gali būti pateikiami ir
% kompaktiniame diske. Priedai taip pat numeruojami ir vadinami. Darbo tekstas
% su priedais susiejamas nuorodomis.
\appendix{Neuroninio tinklo struktūra}

\begin{figure}[H]
    \centering
    \includegraphics[scale=0.5]{img/MLP}
    \caption{Paveikslėlio pavyzdys}
    \label{img:mlp}
\end{figure}


\appendix{Eksperimentinio palyginimo rezultatai}

% tablesgenerator.com - converts calculators (e.g. excel) tables to LaTeX
\begin{table}[H]\footnotesize
  \centering
  \caption{Lentelės pavyzdys}
  {\begin{tabular}{|l|c|c|} \hline
    Algoritmas & $\bar{x}$ & $\sigma^{2}$ \\
    \hline
    Algoritmas A  & 1.6335    & 0.5584       \\
    Algoritmas B  & 1.7395    & 0.5647       \\
    \hline
  \end{tabular}}
  \label{tab:table example}
\end{table}

\end{document}
